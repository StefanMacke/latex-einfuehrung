% In dieser Datei werden einige Standardbefehle gezeigt, um Text zu
% formatieren.

\documentclass{article}
\begin{document}

% Text zentrieren.
\begin{center}
% Der Text wird nach diesem Befehl nun solange groß gesetzt bis eine
% andere Größenangabe folgt.
\Large
The Tragedy of Hamlet, Prince of Denmark\\
% Ab hier ist der Text nun winzig.
\tiny
by William Shakespeare
\end{center}

\noindent % Keine Einrückung des aktuellen Absatzes.
% Man kann die Größenangaben auch nur auf einen kleinen Bereich festlegen, indem
% man die begin/end-Schreibweise nutzt.
% Wichtig: Man kann keine absolute Schriftgröße angeben! Die Angaben sind immer
% relativ zur Größe des allgemeinen Textes ("normalsize").
\begin{tiny}This text is tiny.\end{tiny} \\
\begin{scriptsize}This text is script size.\end{scriptsize} \\
\begin{footnotesize}This text is footnote size.\end{footnotesize} \\
\begin{small}This text is small.\end{small} \\
\begin{normalsize}This text is normal size.\end{normalsize} \\
\begin{large}This text is large.\end{large} \\
\begin{Large}This text is upper-case Large.\end{Large} \\
\begin{LARGE}This text is all-caps LARGE.\end{LARGE} \\
\begin{huge}This text is huge.\end{huge} \\
\begin{Huge}This text is all-caps HUGE.\end{Huge} \\

\noindent
\textit{To be, or not to be: that is the question:}\\ % "italics": kursiv
\textsl{Whether 'tis nobler in the mind to suffer}\\ % "slanted": etwas weniger kursiv
\textbf{The slings and arrows of outrageous fortune,}\\ % "bold face": fett
\texttt{Or to take arms against a sea of troubles,}\\ % "typewriter": nicht-proportionale Schrift
\textsc{And by opposing end them? To die: to sleep;}\\ % "small caps": Kapitälchen
\emph{No more; and by a sleep to say we end}\\ % "emphasized": betont (=kursiv)
The heart-ache and the thousand natural shocks
That flesh is heir to, 'tis a consummation
Devoutly to be wish'd. To die, to sleep;
To sleep: perchance to dream: ay, there's the rub;
For in that sleep of death what dreams may come
When we have shuffled off this mortal coil,
Must give us pause: there's the respect
That makes calamity of so long life;

\newpage

For who would bear the whips and scorns of time,
The oppressor's wrong, the proud man's contumely,
The pangs of despised love, the law's delay,
The insolence of office and the spurns
That patient merit of the unworthy takes,
When he himself might his quietus make
With a bare bodkin? who would fardels bear,
To grunt and sweat under a weary life,
But that the dread of something after death,
The undiscover'd country from whose bourn
No traveller returns, puzzles the will
And makes us rather bear those ills we have
Than fly to others that we know not of?
Thus conscience does make cowards of us all;
And thus the native hue of resolution
Is sicklied o'er with the pale cast of thought,
And enterprises of great pith and moment
With this regard their currents turn awry,
And lose the name of action.--Soft you now!
The fair Ophelia! Nymph, in thy orisons
Be all my sins remember'd.

Get thee to a nunnery: why wouldst thou be a
breeder of sinners? I am myself indifferent honest;
but yet I could accuse me of such things that it
were better my mother had not borne me: I am very
proud, revengeful, ambitious, with more offences at
my beck than I have thoughts to put them in,
imagination to give them shape, or time to act them
in. What should such fellows as I do crawling
between earth and heaven? We are arrant knaves,
all; believe none of us. Go thy ways to a nunnery.
Where's your father?

\end{document}
