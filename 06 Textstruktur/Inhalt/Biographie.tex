% Dieses Kapitel wird mit Hilfe der Befehle \chapter, \section und \subsection
% logisch gegliedert. Anhand der Gliederung wird später automatisch das
% Inhaltsverzeichnis erstellt.

\chapter{Biographie von William Shakespeare}
% Quelle: http://www.william-shakespeare.de/biographie1.html

Ungeachtet der Frage, ob Shakespeare wirklich der Verfasser der ihm
zugeschriebenen Werke war, ist zumindest seine Existenz bezeugt. Leider auch
nicht soviel mehr - was zum einen die altbekannte Frage um die Urheberschaft
seiner Werke aufwirft, zum anderen den Kinoerfolg "`Shakespeare in Love"'
gestattete, sein ganz eigenes Bild des Barden zu entwerfen.

% Man kann den Kapiteln im Inhaltsverzeichnis einen anderen Namen geben.
\section[Die Erzeuger]{Die Eltern}

Obwohl Shakespeares Leben besser bezeugt ist als das vieler seiner
Zeitgenossen, lässt sich seine Biographie nur in groben Umrissen rekonstruieren
-- besonders was die Zeit seiner späten Jugend betrifft. William Shakespeare
wurde laut Kirchregister am 26. April 1564 in Stratford-on-Avon, Warwickshire,
getauft, sein Geburtstag wird heute der Einfachheit halber auf den 23. April
datiert -- ist Shakespeare doch am gleichen Tage des Jahres 1616 verstorben.

\subsection{Der Vater}
Sein Vater, John Shakespeares, war ein angesehener Landwirt und Händler. Er
wurde 1565 zum Stadtrat gewählt, war später Stadtverwalter (eine mit einem
Bürgermeister vergleichbare Position). Aufzeichnungen berichten von einigen
Fehlschlägen in den Geschäften, die zwischenzeitlich wohl zu einer Verarmung
der Familie führten.

\subsection{Die Mutter}
Williams Mutter, Mary Arden of Wilmcote, entstand einem
alten, aber unbedeutenden Adelsgeschlecht und war Erbin eines kleinen Stück
Landes. Entsprechend des damaligen sozialen Gefüges dürfte die Heirat Mary
Ardens für John einen Aufstieg in der lokalen Hierarchie gleichgekommen sein.

\section{Williams Jugend}

Stratford-on-Avon besaß eine Schule von gutem Rufe, die Teilnahme war frei, da
der Unterhalt der Schule vom Bezirk getragen wurde. Diese Tatsache und die
Amtsposition des Vaters lässt vermuten, das William eine gute Ausbildung
erhielt. Diese konzentrierte sich zur damaligen Zeit auf das Studium der
lateinischen Sprache, Dichtung und Geschichte. William besuchte keine
Universität -- ob dies finanzielle Gründe hatte, kann heute nicht mehr
beantwortet werden.

\section{Familienvater}

Im Jahre 1582 -- im Alter von ganzen 18 Jahren -- heiratete er die einige Jahre
ältere Anne Hathaway. Wann genau und wo ist nicht detailliert bekannt,
allerdings registrierte das bischöfliche Sekretariat von Worcester eine
Schuldverschreibung (verbürgt von zwei Stratforder Bauern namens Sandells und
Richardson) als Sicherheit für eine Heiratslizenz von William Shakespeare und
"`Anne Hathaway von Stratford"'. Am 26. Mai 1583 wurde in Stratford Williams
Tochter Susanna, am 2. Februar 1585 seine Zwillinge Hamnet und Judith getauft.
Hamnet, Shakespeares einziger Sohn, verstarb im Alter von 11 Jahren. Seine
Todesursache ist nicht bekannt.
