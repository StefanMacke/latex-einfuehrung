\chapter{Die Schlüsselszene}

\Rolle{HAMLET}
Sein oder Nichtsein; das ist hier die Frage:
Obs edler im Gemüt, die Pfeil und Schleudern
Des wütenden Geschicks erdulden oder,
Sich waffnend gegen eine See von Plagen,
Durch Widerstand sie enden? Sterben -- schlafen --
Nichts weiter! Und zu wissen, daß ein Schlaf
Das Herzweh und die tausend Stöße endet,
Die unsers Fleisches Erbteil, 's ist ein Ziel,
Aufs innigste zu wünschen. Sterben -- schlafen --
Schlafen! Vielleicht auch träumen! Ja, da liegts:
Was in dem Schlaf für Träume kommen mögen,
Wenn wir die irdische Verstrickung lösten,
Das zwingt uns stillzustehn. Das ist die Rücksicht,
Die Elend läßt zu hohen Jahren kommen.
Denn wer ertrüg der Zeiten Spott und Geißel,
Des Mächtigen Druck, des Stolzen Mißhandlungen,
Verschmähter Liebe Pein, des Rechtes Aufschub,
Den Übermut der Ämter und die Schmach,
Die Unwert schweigendem Verdienst erweist,
Wenn er sich selbst in Ruhstand setzen könnte
Mit einer Nadel bloß? Wer trüge Lasten
Und stöhnt' und schwitzte unter Lebensmüh?
Nur daß die Furcht vor etwas nach dem Tod,
Das unentdeckte Land, von des Bezirk
Kein Wandrer wiederkehrt, den Willen irrt,
Daß wir die Übel, die wir haben, lieber
Ertragen als zu unbekannten fliehn.
So macht Bewußtsein Feige aus uns allen;
Der angebornen Farbe der Entschließung
Wird des Gedankens Blässe angekränkelt;
Und Unternehmen, hochgezielt und wertvoll,
Durch diese Rücksicht aus der Bahn gelenkt,
Verlieren so der Handlung Namen. -- Still!
Die reizende Ophelia! -- Nymphe, schließ
In dein Gebet all meine Sünden ein!

\Rolle{OPHELIA}
Mein Prinz, wie geht es Euch seit so viel Tagen?

\Rolle{HAMLET}
Dank untertänigst; wohl, wohl, wohl.

\Rolle{OPHELIA}
Mein Prinz, ich hab von Euch noch Angedenken,
Die ich schon längst begehrt zurückzugeben.
Ich bitt Euch nun, nehmt sie zurück!

\Rolle{HAMLET}
Nein, ich nicht;
Ich gab Euch niemals was.

\Rolle{OPHELIA}
Mein teurer Prinz, Ihr wißt gar wohl, Ihr tatets,
Und Worte süßen Hauchs dabei, die reicher
Die Dinge machten. Da ihr Duft dahin,
Nehmt dies zurück; dem edleren Gemüte
Verarmt die Gabe mit des Gebers Güte.
Hier, gnädger Herr!

\Rolle{HAMLET}
Haha! Seid Ihr tugendhaft?

\Rolle{OPHELIA}
Gnädiger Herr?

\Rolle{HAMLET}
Seid Ihr schön?

\Rolle{OPHELIA}
Was meint Eure Hoheit?

\Rolle{HAMLET}
Daß, wenn Ihr tugendhaft und schön seid, Eure Tugend keinen Verkehr mit Eurer
Schönheit pflegen muß.

\Rolle{OPHELIA}
Könnte Schönheit wohl bessern Umgang haben als mit der Tugend?

\Rolle{HAMLET}
Ja freilich: denn die Macht der Schönheit wird eher die Tugend in eine
Kupplerin verwandeln, als die Kraft der Tugend die Schönheit sich ähnlich
machen kann. Dies war ehedem paradox, aber nun bestätigt es die Zeit. Ich
liebte Euch einst.

\Rolle{OPHELIA}
In der Tat, mein Prinz, Ihr machtet michs glauben.

\Rolle{HAMLET}
Ihr hättet mir nicht glauben sollen, denn Tugend kann sich unserm alten Stamm
nicht so einimpfen, daß wir nicht einen Geschmack von ihm behalten sollten. Ich
liebte Euch nicht.

\Rolle{OPHELIA}
Um so mehr wurde ich betrogen.

\Rolle{HAMLET}
Geh in ein Kloster! Warum wolltest du Sünder zur Welt bringen? Ich bin selbst
leidlich tugendhaft, dennoch könnte ich mich solcher Dinge anklagen, daß es
besser wäre, meine Mutter hätte mich nicht geboren. Ich bin sehr stolz,
rachsüchtig, ehrgeizig; mir stehn mehr Vergehungen zu Dienst, als ich Gedanken
habe, sie zu hegen, Einbildungskraft, ihnen Gestalt zu geben, oder Zeit, sie
auszuführen. Wozu sollen solche Gesellen wie ich zwischen Himmel und Erde
herumkriechen? Wir sind ausgemachte Schurken, alle: trau keinem von uns! Geh
deines Wegs zum Kloster! Wo ist Euer Vater?

\Rolle{OPHELIA}
Zu Hause, gnädiger Herr.

\Rolle{HAMLET}
Laßt die Tür hinter ihm abschließen, damit er den Narren nirgend anders spielt
als in seinem eignen Hause. Leb wohl!

\Rolle{OPHELIA}
O hilf ihm, gütger Himmel!

\Rolle{HAMLET}
Wenn du heiratest, so gebe ich dir diesen Fluch zur Aussteuer: Sei so keusch
wie Eis, so rein wie Schnee, du wirst der Verleumdung nicht entgehn. Geh in ein
Kloster, leb wohl! Oder willst du durchaus heiraten, nimm einen Narren, denn
gescheite Männer wissen allzu gut, was ihr für Ungeheuer aus ihnen macht. In
ein Kloster, geh, und das schleunig! Leb wohl!

\Rolle{OPHELIA}
Himmlische Mächte, stellt ihn wieder her!

\Rolle{HAMLET}
Ich weiß auch von euren Malereien Bescheid, recht gut. Gott hat euch ein
Gesicht gegeben, und ihr macht euch ein anders; ihr schlendert, ihr trippelt,
und ihr lispelt und gebt Gottes Schöpfung verhunzte Namen und gebt eure
Lüsternheit als Einfalt aus. Geht mir, nichts weiter davon, es hat mich toll
gemacht. Ich sage, wir wollen nichts mehr von Heiraten wissen; wer schon
verheiratet ist -- alle außer einem --, soll das Leben behalten; die übrigen
sollen bleiben, wie sie sind. In ein Kloster, geh!
