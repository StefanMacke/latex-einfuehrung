\chapter{Langweiliges Geplänkel}

\Rolle{OPHELIA}
\Ausruf{O welch ein edler Geist ist hier zerstört!}
Des Hofmanns Auge, des Gelehrten Zunge,
Des Kriegers Arm, des Staates Blum und Hoffnung,
Der Sitte Spiegel und der Bildung Muster,
\Ausruf{Das Merkziel der Betrachter: ganz, ganz hin!}
Und ich, der Fraun elendeste und ärmste,
Die seiner Schwüre Honig sog, ich sehe
Die edle, hochgebietende Vernunft
Mißtönend wie verstimmte Glocken jetzt,
Dies hohe Bild, die Züge blühnder Jugend,
Durch Überschwang zerrüttet: Weh mir, wehe,
Daß ich sah, was ich sah, und sehe, was ich sehe.
Der König und Polonius treten wieder vor.

\Rolle{KÖNIG}
Aus Liebe? Nein, sein Hang geht dahin nicht,
Und was er sprach, obwohl ein wenig wüst,
War nicht wie Wahnsinn. Ihm ist was im Gemüt,
Worüber seine Schwermut brütend sitzt,
Und, wie ich sorge, wird die Ausgeburt
Gefährlich sein. Um dem zuvorzukommen,
Hab ichs mit schleuniger Entschließung
So vorgesehn: Er soll in Eil nach England,
Den Rückstand des Tributes einzufordern.
Vielleicht vertreibt die See, die neuen Länder
Samt wechselvollen Gegenständen ihm
Dies Etwas, das in seinem Herzen steckt,
Worauf sein Kopf, beständig hinarbeitend,
Ihn so sich selbst entzieht. Was meint Ihr dazu?

\Rolle{POLONIUS}
Es wird ihm wohltun, aber dennoch glaub ich,
Der Ursprung und Beginn von seinem Gram
Sei unerhörte Liebe. - Nun, Ophelia?
Ihr braucht uns nicht zu melden, was der Prinz
Gesagt; wir hörten alles. - Gnädger Herr,
Tut nach Gefallen; aber dünkts Euch gut,
So laßt doch seine königliche Mutter
Ihn nach dem Schauspiel ganz allein ersuchen,
Sein Leid ihr kundzutun; sie mag nur rund
Heraus ihn fragen. Ich, wenns Euch beliebt,
Stell ins Gehör der Unterredung mich.
Wenn sie es nicht herausbringt, schickt ihn dann
Nach England oder schließt ihn irgendwo
Nach Eurer Weisheit ein.

\Rolle{KÖNIG}
Es soll geschehn;
Wahnsinn bei Großen darf nicht ohne Wache gehn.
